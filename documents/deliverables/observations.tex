
\section{AiPlayer}\label{AiPlayer}
  

The specification for AiPlayer dictates that the AiPlayer should submit an empty string whenever it is not allowed to make a move.
From the specification for \textbf{AiPlayer}: 

\begin{quotation}
Returns: the move to make. Should return an empty string if this player cannot make a move (because the opponent's move ends the game, 
or for some other reason).
\end{quotation} 

The fulfillment of this requirement brings several issues into consideration. The choice of overall architecture, in 
which players are exclusively asked to make a move when they are required to do so, is incompatible with the nature of 
this specification. An implementation that could trivially support it would require direct involvement in the 
turn management and knowledge of the specific game rules regarding termination and conditions such as \emph{stalemate}. 
Such a direct involvement with the game requires a significant amount of coupling that can hinder the possibility to implement 
further players or extensions to the game. Therefore, \emph{Pawned} uses a notification system in which players are asked
to submit a move only when it is necessary for them to do so. In order to fulfill the specification, \emph{Pawned}'s implementation
of \textbf{AiPlayer} acts as a representative for the other player in the tournament, playing in the \textbf{Controller} against
the actual \textbf{AIPlayer}. Since \textbf{AiPlayer} is not asked to submit empty strings by \emph{Pawned}, \textbf{AiPlayer}
uses a queuing system for the actual \textbf{AIPlayer}'s moves as they are notified from the system, in order 
to detect and send these through it's \texttt{makeMove(move, time)} method. 

For example, in a game such as \emph{N-move} chess, where each player has to play N moves during his turns, the turn delegation becomes 
problematic. In this case, there are two options either allow the user to input all of his moves in a single string after he receives a 
\texttt{makeMove(move, time)} call, or query him N times for a move. In the first case, the other player will be forced to submit twenty
empty strings to the referee, thereby forcing the player to constantly report to the referee when it has no necessity to do so. Not only 
does this result in a significant design flaw, but could also have performance repercussions in cases such as calls to the players
via the Internet, such as \emph{RMI} invocations. On the other hand, if all the moves are input at once, then the sequential nature of the
\texttt{makeMove(move,time)} calls is broken, making the return of an empty string seem arbitrary. 

In addition, the choice to request an empty string for invalid moves makes the application more prone to failure and makes it harder to
recover from erroneous input. Since a  player can return an invalid move when it is not his turn, \emph{makeMove(move,time)}, there
is a chance that the player will submit some type of ply. Further, there is no way to determine the time remaining for the other player,
so the player will be unable to submit an empty string when the game has finished as a result of time depletion, thus making the 
player unable to completely determine whether the game is still being played or not. 

it is apparent that due to this design choice, several computations are bound to be repeated. Since the referee has to verify whether a 
move is valid or not, then he needs to determine whose turn it is, in order to predict whether a given player should have returned an 
empty string. On the other side, each player is required to carry out this computation, since it has to determine whether it has to return 
the empty string or not. 

\section{TextUI} 
 
TextUI does not make use of the asynchronous information messaging system provided by \emph{Pawned}. In fact, to an extent, 
the requirements for TextUI and the expected functionality of \texttt{Controller} are incompatible. For example, 
the TextUI is expected to be able to control the artificial intelligence player's movements, whereas the \texttt{Controller}
assumes the role of ply retriever in the execution environment. For this reason, TextUI was implemented at the \texttt{Game}
level, and not the \texttt{Controller} level. 

The TextUI was extensively used as a testing mechanism for \texttt{Game}, since it is a wrapper for it, without the
additional functionality provided by the \texttt{Controller}. However, the TextUI lacks the extensibility supported
by the rest of the execution environment, since it is tightly coupled with assumptions regarding its role in the
game process. For example, consider the problem of allowing players compete against each other over Internet. Since 
the command line interface heavily relies on synchronous communication and 

Further, the requirement that the players themselves submit the time they took to submit the move makes it hard to possess 
a reliable time measurement, since an accurate measuring cannot be achieved. On the other hand, \texttt{Controller} could easily
allow an implementation of such an addition, by implementing a particular version of \texttt{Player} which is capable of 
communicating across the Internet. 