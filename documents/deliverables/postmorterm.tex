\section{Evaluation}
\section{Lessons}

\section{Known bugs and limitations}
	\subsection{Captured pieces} 
	The definition of a captured piece in a board is the following: When a ply is executed on a board, if a piece that 
	was inside a board before the ply was executed is no longer in the board when the ply finished, then the ply is
	considered `captured'. However, this interpretation cannot fully incorporate particular requirements for the games.
	
	For example, in chess, it is not necessarily true that a pawn that has been coronated is `captured'. However,
	according to the definition, this piece would actually be captured. Further, some games do not consider the notion
	of captured pieces (e.g. connect four), so such functionality is not particularly useful. 
	
	A possible solution for this problem could be to include this information in the set of objects of type
	\texttt{GameMessage}. However, further analysis of this problem is required to provide a full solution.
	
	\subsection{GUI Threading}
	As mentioned before, the GUI behaves poorly under stress testing. Although the situation in which players
	move almost instantaneously is very unlikely, this situation could bring undefined results in a slow computer
	under heavy load. However, the GUI must be inspected to ensure proper thread management. 
	
	\subsection{AI Player}
	Currently, the AI Player uses a minimax algorithm in order to fulfill the parallelizability requirement. 
	Because this algorithm is very easily transofmred into a concurrent algorithm, performance increases almost
	by about as much as the increase in the number of processors. 
	
	However, in doing so, the AI Player fails to use any optimization for the minimax algorithm, such as 
	$\alpha \beta$ pruning. Since $\alpha \beta$ pruning requires a sequential depth-first search on the 
	game tree, it is hard to obtain satisfactory results while parallelizing, since a naive parallelization 
	would require computation of more nodes than what would be required in a sequential algorithm. 
	
	In order to produce a satisfactory AI Player, the AI Player should build a transposition table that is
	concurently available and modifiable by all the concurrent threads. However, time constraints did not 
	allow the implementation of such a system. Although the minimax algorithm does fulfill the requirement
	of using the power of parallel processors, its expected performance is poor. 


\subsection{A final note on extensibility}\label{final-extensibility}

\emph{Pawned} constanatly attempts to provide an extensible environment for game execution. The controller
will always  ensure that the game is played following the rules from the rule set. 
However, to do so, the objects that hold a reference
to the instantiated controller keep an implicit promise not to modify the board in play. However, for a 
publishable API, this behavior might be very dangerous. For example, consider an Internet enabled version of the controller, 
such that external developers can create \texttt{GameObservers} and \texttt{Players}. Malicious attacks would be easily 
carried out by modifying the board in play with arbitrary plies, thus rendering the game unplayable. 

However, this situation is easily solvable by cloning the board whenever it is requested, instead
of providing the player with the actual board. This feature was not included on this version of \emph{Pawned} since
it can be guaranteed that none of the UIs will ever modify the board directly. 