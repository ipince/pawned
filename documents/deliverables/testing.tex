\section{Strategy}

	\subsection{Testing of \texttt{engine.adt}}
	Testing \texttt{engine.adt} posed the problem that the majority of its classes are not fully implemented (some methods
	are abstract). In order to solve this problem, basic types of each abstract class were created and those classes 
	were unit tested. 
	
	\texttt{CountingPiece} and \texttt{SimplePiece} are two pieces that were developed to test board operations and functionality 
	of the board's messaging systems. Special attention was paid to ensuring that the order of the messages behaved
	in a manner consistent with the specification, and also that the pieces were successfully added, removed, and cloned
	within the board, ensuring that the specifications are met on all the different possible states of a board position
	(unusable, empty, and occupied).  
	
	In order to provide a board concrete class, \texttt{UselessBoard} and \texttt{LineBoard}, 0 and 1 dimensional boards, 
	respectively, were created in order to place the implemented pieces for testing. 	
	
	\subsection{Testing of \texttt{RectangularBoard}} In order to test the rectangular board, a test driver was developed.
	This test driver is capable of creating boards and adding pieces to it. The addition and removal of pieces was carried 
	out under different circumpstances (on unusable, empty, and unoccupied cells) and the behavior was tested for 
	compliance with the specification. 
	
	\subsection{Testing of \texttt{StdACRuleSet}} Several unit tests were implemented for each piece in \emph{Standard 
												6.170 Antichess} and \emph{6.170 Antichess with Castling}. A rectangular board was set up with 
												different configurations of pieces, and each piece was queried for its valid plies, 
												comparing them against their expected results. 
												
												Additionally, the ruleset's global filtering was tested using a similar mechanism.
												
	\subsection{Testing of \texttt{controller.*}}The components of the controller package (StopWatch and TurnCycle) 
						were tested separately. Controller was tested along with the GUI, both in regular and in stress 
						testing. The XML Factories were tested using unit tests as well. 

	\subsection{Testing of \texttt{Game} and \texttt{TextUI}} \texttt{TextUI} and \texttt{Game} were tested together. 
	Since \texttt{TextUI} is a wrapper directly built on top of \texttt{Game}. 
	
	\subsection{Testing of the GUI} Since the development of the Graphical User Interface could not wait until 
	the remainder of the game was finished, a temporary controller lookalike: \texttt{Kontroller}, was developed 
	to simulate calls by controller to the GUI. This top-down approach to the GUI development allowed for an early
	set of initial tests on the user interface.
	
		The GraphicUI tests consist of hand made tests carefully chosen and recorded
		to prove the functionality of \emph{Pawned}. An scenario will be described in 
		detail, and then we will check that the GraphicUI behaves as expected.
				
		\subsection{Opening \textit{Pawned}} 
		
		Three menus are enabled, with the following submenus: 
		\begin{itemize}
			\item File: \textit{New Game}, \textit{Save Game}, \textit{Load Game}, and 
									\textit{Quit}
			\item Game: \textit{End Game} and \textit{Display Options}  
			\item Help: \textit{User Manual} and \textit{About}
		\end{itemize}
		
		The Save Game and End Game options in the menus are disabled.
		Each submenu except \textit{Display Options} will open a new window. 
		\textit{Display Options} contains a submenu with three options that can be 
		checked individualy (Highlight Movable Pieces, Hightlight Possible Moves, 
		Highlight Last Move). These options are disable from user selection before 
		starting a game.
		
		There is a ToolBar with one button for each of the following options:
		New Game, Load Game, Save Game, and End Game. The Save Game and End Game
		buttons are disabled.
		
		The window can be minimized and closed using the small buttons in the
		upper corner of the window. Maximize is disabled. The main panel 
		should be empty.
		
		The following hot keys are enabled:
				\begin{itemize}
					\item New Game - Ctrl + N
					\item Load Game - Ctrl + O
					\item User Manual - Ctrl + U
					\item Quit - Ctrl + Q
				\end{itemize}		
		 
		 
		\subsubsection{New Game}
			The New Game window can be accessed by either the New Game option in 
			the File menu of the button in the toolbar. For each of the players
			the New Game window has:
			\begin{itemize}
					\item Text field to input the player's name (max 8 char)
					\item Radio buttons to select between Human or Computer
					\item Combo box to select the intelligence level of the Computer
								player (Baby, Kid, Adolescence)
					\item Checkbox to select Timed game 
					\item Spinners to input the time (one spinner for minutes and 
								another for seconds)
				\end{itemize}
			If the Computer radio button is selected, the computer level combo box is enabled.
			Otherwise it is disabled. If the Timed checkbox is selected, then the 
			Time spinners are enabled and can receive an input. Otherwise they 
			are disabled. The spinners are set to receive only a number from 0 to 60 for 
			minutes and from 0 to 59 for seconds.
			
			The New Game window also contains a combo box where the user can select the 
			type of game desired (Standard Antichess, EnCastle Antichess, and Connect N).
			When this window is opened, the default game selected is Standard Antichess and
			two pawn images are displayed, one pawn of each color, to identify the players. 
			If EnCastle Antichess is selected everything stays the same. If Connect N is 
			selected, the pawns change to a red chip and a black chip. A spinner appears to
			the right of the game type combo where the user can select the value \textsl{n} 
			for the Connect N game. The spinner will let you to enter values from 2 to 7.
			
			\subsection{Load Game}
			
			This window is very similar to the New Game window. The differences are:
				\begin{itemize}
					\item There is a read-only text field and a Browse button. The Browse
								button opens a file dialog window that is set to filter XML files.
								The text field is updated after the file is selected from the file
								dialog. 
					\item The Timed checkbox is disable, along with the Time spinner fields.
					\item The game type combo box is disabled.
				\end{itemize}
				
			After selecting a file in the File dialog, the Load Window will refresh and
			show some information about the file selected (assuming the file is valid).
			If the saved game was untimed, then the Timed checkboxs will be unchecked and
			the Time fields disabled. If one of the players was set up with limited time, 
			then its Timed checkbox will be checked and the Time fields will display the
			remaining time it has for playing. If both players were timed, then this
			information will be updated for both. At this point \emph{Pawned} will allow 
			the user to change these options. 
			
			The game type combo box will also be updated with the type of game of the 
			saved file. However, this field will always remain disabled. The spinner for
			selecting the \textsl{n} value for Connect N also remains diabled, but it
			refreshes an shows the n value of a saved Connect N game.
			
			Cases to test for each type of game:
			\begin{itemize}
					\item Untimed game
					\item Only one player untimed.
					\item Two timed players.
				\end{itemize}
			
			Other things to test:			
			\begin{itemize}
					\item If a non existing file is chosen in the File dialog, display an 
								error message box.
					\item If 'OK' is clicked without selecting a file display an error message
								box.
					\item If a corrupted file is tried to be loaded, display an error message box.
				\end{itemize}


			\subsection{Starting new games and changing from game type to game type}
			
			\emph{Pawned} allows you to change from two game environments that are further 
			apart from each other than two variations of Chess. The GraphicUI was tested to 
			assure the ease of this change.		
	
	\subsection{Controller-GUI stress testing}
		In order to test a correct threading and concurrency implementation, several matches between extremely fast
		random players were run, which return their moves immediately after they are queried. At the controller level, 
		it was tested that 
		\begin{itemize}
			\item A controller is successfully initialized, both starting a new game or loading an XML game,
					  even when the players will immediately respond to a request. 
			\item A game can be saved when players are engaged in a very fast gaming activity.
			\item A controller successfully terminates a game when players are engaged in very fast playing.
			\item The sets of captured pieces and plies do not become inconsistent when the players 
						are exercising moves very fast.
		\end{itemize}
		
		The previous items were run with delays on different parts of the code (e.g. the \texttt{RunCycle} 
		in charge of turn management, the \texttt{inform(Ply ply)} method in \texttt{Controller}, in charge
		of notifying the players of the plies as they are executed by the players. 
		
		The \texttt{StopWatch} objects were tested for accuracy to within $5$ ms under regular and heavy 
		CPU load. Additionally, tests were run to ensure that the pauses and starts are responsive to within 
		1 millisecond in a \emph{Centrino Duo, 1.66 GHz}, as well as a \emph{Centrino, 1.5 GHz} machine. 
		Finally, a stopwatch test on an interval of \texttt{Integer.MAXVALUE} was carried out to test 
		border cases for the stop watch. 
		
		All these tests were successful. 
		
		In the same way, the GUI was tested under similar conditions, in order to ensure appropriate behavior, 
		even under extreme conditions. Games were saved and loaded one after another, and while games with the
		fast random player were being carried out. Although several tests were successful, some race conditions
		were detected, as well as some deadlocks under certain stringent conditions. For example, when a game is
		being played by extremely fast random players, and the termination condition is launched exactly at the time
		in which a person is choosing a file in the second dialog of the GUI, the main window comes to a halt. 
		This situation is solved whenever the players take longer than 10 ms to make a move. 
		
		