

		Main is an interactive loop between a human and a computer. The \texttt{TextUI} should 
		wait for user input. Valid user input is in the form of the commands shown in 
		the requires clause. Parameters to commands are enclosed in square brackets. 
		Outputs as a result of a command should be terminated with a newline. If 
		multiple lines are output, the last line should also be terminated with a new 
		line.

		When main is executed with no command-line arguments, supported commands are:

		\textbf{StartNewGame} \texttt{[player] [time] [player] [time]}
    
    Each player is denoted as 'human' or 'computer.' The times specified are the 
    initial 'time remaining' for the player, in milliseconds. If a time argument is 
    zero, then the playing time is unlimited. The first player is white and the second 
    player is black. For example, StartNewGame computer 60000 human 180000 should start 
    a new game with the computer playing as white and the human playing as black. The 
    computer will have 1 minute to make all of its moves, and the human will have 3 
    minutes. The system should output New game started on its own line.

		\textbf{SaveGame} \texttt{[filename]}
    
    The system should save the game to the given filename and report Game saved on 
    its own line.

		\textbf{LoadGame} \texttt{[filename]}
    
    The system should load the game from the given filename. Once the files is loaded, 
    print Game loaded on its own line. You should report Corrupt file if the file does 
    not have a correct format. We will not require you to determine if the board is 
    legal. If no game is currently in progress from a previously executed StartNewGame 
    or LoadGame command, then assume a human-human game.

		\textbf{GetNextMove}
    
    If this command is called during a human player's turn, the command prints Human 
    turn on its own line. If this command is called during a machine player's turn, 
    print on its own line the next move it believes to be the best. The printed move 
    should be in the 'standard string format' described in the assignment. The time 
    aken to compute the move should be subtracted from the computer player's game clock. 
    If called repeatedly, this should return the same move over and over without further 
    decrementing the computer's time remaining.

		\textbf{MakeNextMove}
    
    If it is a human player's turn, the system should print Please specify human move 
    on its own line. If it is the computer player's turn, and GetNextMove has not yet 
    been called on this turn, then the system should print First GetNextMove. Otherwise, 
    the system performs the move that GetNextMove would return. 
		
		\textbf{MakeMove} \texttt{[move] [time]}
    
    Perform the move specified by the move string, in the 'standard string format' described 
 	  in the assignment. The time parameter is specified in milliseconds. This command should 
    only be used by a Human Player. If it is used during a computer player's turn, nothing 
  	will happen to the game state and no response should be printed. If the move is not legal,
    the system should print, on its own line, Illegal move and not perform the move. If 
    the move is legal, the system should perform the move, decrement the player's time by 
    the amount given, and print the move performed, in proper format, on its own line. If 
    the player's time is unlimited, then the time argument is ignored (but must still be present).

		\textbf{PrintBoard}
    
    System should print the current 'state' of the game to the screen using the same format 
    as if it were being saved to a file. The output should end with (at least one) a newline.

		\textbf{IsLegalMove} \texttt{[move]}
    
    System should print, on its own line, either 'legal' or 'illegal' to specify if the move 
    is a legal next move.

		\textbf{PrintAllMoves}
    System should, in alphanumeric order, print all legal moves for the next player. Each move 
    should each appear on its own line.
		
		\textbf{GetTime} \texttt{[player]}
    
    On its own line, the system should print the time remaining in milliseconds for the player 
    specified. For example GetTime white, should print 3000 to indicate 3 seconds left for the 
    white player. If the time for the player is unlimited, the system should print 'unlimited'.

		\textbf{QuitGame}
    
    Prints (on its own line) Exiting game and terminates the present game and application. 
    QuitGame cannot be the first command.

		
		For each command other than GetNextMove the specified behavior completes within 10 seconds. 
		GetNextMove may take no more than ten seconds more than the player's time remaining to 
		complete; if it exceeds the player's time remaining it must report a human victory in the 
		format described below.

		If the user input does not match one of these commands, output Input 
		error alone on one line. Also, the first valid command entered must 
		be either StartNewGame or LoadGame, or else Input error is printed.

		When a player has won the game, output on its own line: [Player color] Player has won. For 
		example, if the black player has won, output: Black Player has won. At this point, you can
		 assume that the Antichess program has just been started and therefore, you only need to 
		support the subset of commands.

		The behavior of the TextUI is unspecified when main is run with one or more command-line arguments.
		
